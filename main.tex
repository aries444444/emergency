\documentclass[12pt,a4paper]{article}
\usepackage[margin=1in]{geometry} % 页边距1英寸
\usepackage{amsmath,amssymb,amsthm} % 数学公式包
\usepackage{ctex} % 中文支持
\usepackage{fancyhdr} % 页眉页脚
\usepackage{enumitem} % 列表格式
\usepackage{setspace} % 行距设置

% 字体设置:中文宋体,英文Times New Roman
\setCJKmainfont{SimSun} % 中文主体字体
\setmainfont{Times New Roman} % 英文主体字体

% 行距设置:1.5倍行距
\linespread{1.5}

% 页眉页脚设置
\pagestyle{fancy}
\fancyhf{} % 清空默认页眉页脚
% 页眉:左-标题,右-章节名
\lhead{\bfseries Calculus Final Exam Review}
\rhead{\bfseries \leftmark}
% 页脚:中-版权信息,右-页码
\cfoot{\small © The XJTLU Math Club – All rights reserved}
\rfoot{\small Page \thepage}
% 页眉页脚线条
\renewcommand{\headrulewidth}{0.5pt}
\renewcommand{\footrulewidth}{0.3pt}

% 定理/定义格式(保持原文风格)
\theoremstyle{plain}
\newtheorem{theorem}{定理}[section]
\newtheorem{definition}{定义}[section]

\begin{document}

% 仅保留5.3到8.1章节,严格遵循原文内容,不增不减
\section*{5.3 The First Fundamental Theorem of Calculus}
\addcontentsline{toc}{section}{5.3 微积分第一基本定理}

\subsection*{Theorem A First Fundamental Theorem of Calculus(微积分第一定理)}
设 \( f \) 是闭区间 \([a, b]\) 上的连续函数,\( x \) 是 \((a, b)\) 内一点,则
\[
\frac{d}{dx} \int_{a}^{x} f(t) dt = f(x)
\]
(Let \( f \) be continuous on the closed interval \([a, b]\) and let \( x \) be a (variable) point in \((a, b)\). Then \(\frac{d}{dx} \int_{a}^{x} f(t) dt = f(x)\).)

\subsection*{Theorem B Comparison Property(保不等式性)}
如果函数 \( f \) 和 \( g \) 在 \([a, b]\) 上可积,且对于所有在 \([a, b]\) 内的 \( x \) 都有 \( f(x) \leq g(x) \),则
\[
\int_{a}^{b} f(x) dx \leq \int_{a}^{b} g(x) dx
\]
(If \( f \) and \( g \) are integrable on \([a, b]\) and if \( f(x) \leq g(x) \) for all \( x \) in \([a, b]\), then \(\int_{a}^{b} f(x) dx \leq \int_{a}^{b} g(x) dx\).)

用非正式的描述性语言来说就是定积分具有保持不等关系。(In informal but descriptive language, we say that the definite integral preserves inequalities.)

\subsection*{Theorem C Boundedness Property(有界性)}
如果 \( f \) 在区间 \([a, b]\) 上可积,而且对于所有在 \([a, b]\) 内的 \( x \),有 \( m \leq f(x) \leq M \),那么
\[
m(b - a) \leq \int_{a}^{b} f(x) dx \leq M(b - a)
\]
(If \( f \) is integrable on \([a, b]\) and \( m \leq f(x) \leq M \) for all \( x \) in \([a, b]\), then \( m(b - a) \leq \int_{a}^{b} f(x) dx \leq M(b - a) \).)

\subsection*{Theorem D Linearity of the Definite Integral(定积分的线性性质)}
如果 \( f \) 和 \( g \) 在区间 \([a, b]\) 上可积,且 \( k \) 是一个常数,则 \( kf \) 与 \( f + g \) 可积,且有
\[
\int_{a}^{b} kf(x) dx = k \int_{a}^{b} f(x) dx
\]
\[
\int_{a}^{b} [f(x) + g(x)] dx = \int_{a}^{b} f(x) dx + \int_{a}^{b} g(x) dx
\]
(Suppose that \( f \) and \( g \) are integrable on \([a, b]\) and that \( k \) is a constant. Then \( kf \) and \( f + g \) are integrable and the above equalities hold.)

\subsection*{Position(位移)}
把 \( F(a) = \int_{0}^{a} v(t) dt \) 定义为物体在时间 \( a \) 时的位移。(Let \( F(a) = \int_{0}^{a} v(t) dt \) denote the position of the object at time \( a \).)

---

\section*{5.4 The Second Fundamental Theorem of Calculus and the Method of Substitution}
\addcontentsline{toc}{section}{5.4 微积分第二基本定理及换元法}

\subsection*{Theorem A Second Fundamental Theorem of Calculus(微积分第二基本定理)}
假设 \( f \) 在 \([a, b]\) 上连续(因此可积),并假设 \( F \) 为 \( f \) 在 \([a, b]\) 上的任一原函数,则
\[
\int_{a}^{b} f(x) dx = F(b) - F(a)
\]
(方便记法:\(\int_{a}^{b} f(x) dx = \left[ \int f(x) dx \right]_{a}^{b}\))
(Let \( f \) be continuous (hence integrable) on \([a, b]\), and let \( F \) be any antiderivative of \( f \) on \([a, b]\). Then \(\int_{a}^{b} f(x) dx = F(b) - F(a)\) (convenient writing: \(\int_{a}^{b} f(x) dx = \left[ \int f(x) dx \right]_{a}^{b}\)).)

\subsection*{Theorem B Substitution Rule for Indefinite Integrals(不定积分的换元积分法)}
令 \( g \) 为一个可导函数且函数 \( f \) 为函数 \( F \) 的导数,则
\[
\int f(g(x)) g'(x) dx = F(g(x)) + C
\]
(Let \( g \) be a differentiable function and suppose that \( F \) is an antiderivative of \( f \). Then \(\int f(g(x)) g'(x) dx = F(g(x)) + C\).)

\subsection*{Theorem C Substitution Rule for Definite Integrals(定积分的换元法则)}
设函数 \( g \) 在区间 \([a, b]\) 上有一个连续的导数,且 \( f \) 在 \( g \) 的值域上连续,则有
\[
\int_{a}^{b} f(g(x)) g'(x) dx = \int_{g(a)}^{g(b)} f(u) du
\]
其中 \( u = g(x) \)。
(Let \( g \) have a continuous derivative on \([a, b]\), and let \( f \) be continuous on the range of \( g \). Then \(\int_{a}^{b} f(g(x)) g'(x) dx = \int_{g(a)}^{g(b)} f(u) du\) where \( u = g(x) \).)

\subsection*{Accumulated Rate of change(累积的变化率)}
see p309 example16(见书本P309例16)

---

\section*{5.5 The Mean Value Theorem for Integrals and the Use of Symmetry}
\addcontentsline{toc}{section}{5.5 积分中值定理和对称性的应用}

\subsection*{Definition Average Value of a Function(函数的平均值)}
如果函数 \( f \) 在区间 \([a, b]\) 上可积,则其在 \([a, b]\) 上的平均值为
\[
f_{\text{avg}} = \frac{1}{b - a} \int_{a}^{b} f(x) dx
\]
(If \( f \) is integrable on the interval \([a, b]\), then the average value of \( f \) on \([a, b]\) is \( f_{\text{avg}} = \frac{1}{b - a} \int_{a}^{b} f(x) dx \).)

\subsection*{Theorem A Mean Value Theorem for Integrals(积分中值定理)}
如果函数 \( f \) 在区间 \([a, b]\) 上连续,在 \( a \) 和 \( b \) 之间必存在点 \( c \) 满足
\[
f(c) = \frac{1}{b - a} \int_{a}^{b} f(x) dx
\]
或等价地,\(\int_{a}^{b} f(x) dx = f(c)(b - a)\)
(If \( f \) is continuous on \([a, b]\), then there is a number \( c \) between \( a \) and \( b \) such that \( f(c) = \frac{1}{b - a} \int_{a}^{b} f(x) dx \) or equivalently, \(\int_{a}^{b} f(x) dx = f(c)(b - a)\).)

\subsection*{Theorem B Symmetry Theorem(对称性定理)}
如果 \( f \) 是个偶函数,则
\[
\int_{-a}^{a} f(x) dx = 2 \int_{0}^{a} f(x) dx
\]
如果 \( f \) 是个奇函数,则
\[
\int_{-a}^{a} f(x) dx = 0
\]
(If \( f \) is an even function, then \(\int_{-a}^{a} f(x) dx = 2 \int_{0}^{a} f(x) dx\); If \( f \) is an odd function, then \(\int_{-a}^{a} f(x) dx = 0\).)

\subsection*{Theorem C}
如果 \( f \) 是以 \( p \) 为周期的函数,则
\[
\int_{a}^{a + p} f(x) dx = \int_{0}^{p} f(x) dx
\]
对任意实数 \( a \) 成立。
(If \( f \) is periodic with period \( p \), then \(\int_{a}^{a + p} f(x) dx = \int_{0}^{p} f(x) dx\) for any real number \( a \).)

---

\section*{6.1 The Area of a Plane Region}
\addcontentsline{toc}{section}{6.1 平面区域的面积}

\subsection*{A Region above or below the x-axis(在x轴以上或以下的区域)}
猜想由 \( y = f(x) \),\( x = a \),\( x = b \),和 \( y = 0 \) 围成的区域 \( R \),它的面积 \( A(R) \) 为
\[
A(R) = \int_{a}^{b} |f(x)| dx
\]
如果是 \( x \) 轴下方的区域,则是对应的 \( x \) 轴上方区域面积的相反数。
(Supposed that the region \( R \) bounded by the graphs of \( y = f(x) \), \( x = a \), \( x = b \), and \( y = 0 \). Its area \( A(R) \) is given by \( A(R) = \int_{a}^{b} |f(x)| dx \). If the region is below the x-axis, it is just the negative of the area of the region above x-axis.)

\subsection*{A Region Between Two Curves(两条曲线之间的区域)}
\paragraph{method(方法)}
slice, approximate, integrate(分割、近似、积分)(它分为五个详细步骤)
\begin{enumerate}[label=Step \arabic*:]
    \item Sketch the region.(作出区域图形)
    \item Slice it into thin pieces (strips); label a typical piece.(将这个区域切割成一个个小的片状或带状部分,然后标记其中一个特定部分)
    \item Approximate the area of this typical piece as if it were a rectangle.(将这个标记的部分近似看成为一个长方形)
    \item Add up the approximations to the areas of the pieces.(将区域中所有近似部分累加)
    \item Take the limit as the width of the pieces approaches zero, thus getting a definite integral.(对累计的结果取极限,让每一小部分的宽度趋近于0,从而得到了一个定积分表达式)
\end{enumerate}

\paragraph{solution(解法)}
设 \( a \)、\( b \) 是曲线的交点,若 \( f(x) \geq g(x) \) 在 \([a, b]\) 上恒成立,则面积
\[
A = \int_{a}^{b} [f(x) - g(x)] dx
\]
(a、b are the intersection point of the curve; If \( f(x) \geq g(x) \) for all \( x \in [a, b] \), then the area \( A = \int_{a}^{b} [f(x) - g(x)] dx \).)

\subsection*{distance and displacement(距离和位移)}
- Displacement(位移量):\(\int_{t_1}^{t_2} v(t) dt\)
- The total distance(总距离):\(\int_{t_1}^{t_2} |v(t)| dt\)

---

\section*{6.2 Volumes of Solids: Slabs, Disks, Washers}
\addcontentsline{toc}{section}{6.2 立体的体积:薄片模型、圆盘模型、圆环模型}

\subsection*{Solids of Revolution: Method of Disks(旋转立体的体积:圆盘法)}
\paragraph{concept(概念)}
当一个平面区域绕着这个平面内的一条固定的直线旋转时,产生了一个旋转立体,这条固定的直线叫作这个旋转立体的轴线。可以用定积分表示这些旋转体的体积。
(When a plane region, lying entirely on one side of a fixed line in its plane, is revolved about that line, it generates a solid of revolution. The fixed line is called the axis of the solid of revolution. It is possible to represent the volume as a definite integral.)

\paragraph{solution(解)}
look at p343 example1、2(见书本P343例1、例2)
- 若区域由 \( y = f(x) \)、\( x = a \)、\( x = b \)、\( y = 0 \) 围成,绕x轴旋转,体积 \( V \) 为
\[
V = \pi \int_{a}^{b} [f(x)]^2 dx
\]

\subsection*{Method of Washers(圆环法)}
\paragraph{concept(概念)}
切割一个旋转体得到中间有洞的圆盘。
(slicing a solid of revolution results in disks with holes in the middle.)

\paragraph{solution(解)}
look at p344 example3、4(见书本P344例3、例4)
- 若区域由 \( y = f(x) \)、\( y = g(x) \)、\( x = a \)、\( x = b \)(\( f(x) \geq g(x) \geq 0 \))围成,绕x轴旋转,体积 \( V \) 为
\[
V = \pi \int_{a}^{b} \left( [f(x)]^2 - [g(x)]^2 \right) dx
\]

---

\section*{6.3 Volumes of Solids of Revolution: Shells}
\addcontentsline{toc}{section}{6.3 旋转体的体积:薄壳法}

\subsection*{the method of cylindrical shells(柱形壳法)}
\paragraph{concept(概念)}
一个柱形壳是两个同心的圆柱体所围成的,如果内径 \( r_1 \),外径 \( r_2 \),高是 \( h \),则体积为
\[
V = 2\pi \cdot (\text{average radius}) \cdot (\text{height}) \cdot (\text{thickness}) = 2\pi rh\Delta r
\]
其中表达式 \( (r_1 + r_2)/2 \) 用 \( r \) 表示,是 \( r_1 \) 和 \( r_2 \) 的平均值(平均半径)。
(A cylindrical shell is a solid bounded by two concentric right circular cylinders. If the inner radius is \( r_1 \), the outer radius is \( r_2 \), and the height is \( h \), then its volume is given by \( V = 2\pi \cdot (\text{average radius}) \cdot (\text{height}) \cdot (\text{thickness}) = 2\pi rh\Delta r \), where \( r = (r_1 + r_2)/2 \) is the average radius.)

\paragraph{shell method(薄壳法)}
look at p349-350 example1、2(见书本P349-350例1、例2)
- 若区域由 \( y = f(x) \)、\( x = a \)、\( x = b \)、\( y = 0 \)(\( f(x) \geq 0 \))围成,绕y轴旋转,体积 \( V \) 为
\[
V = 2\pi \int_{a}^{b} x f(x) dx
\]

---

\section*{6.4 Length of a Plane Curve}
\addcontentsline{toc}{section}{6.4 平面曲线的弧长}

\subsection*{definition(定义)}
设一平面曲线的参数方程为 \( x = f(t) \),\( y = g(t) \),\( a \leq t \leq b \),\( f' \) 和 \( g' \) 存在且在定义域 \([a, b]\) 上连续,\( f'(t) \) 和 \( g'(t) \) 在 \([a, b]\) 上不同时为0,则称该曲线是光滑的。
(A plane curve is smooth if it is determined by a pair of parametric equations \( x = f(t) \), \( y = g(t) \), \( a \leq t \leq b \), where \( f' \) and \( g' \) exist and are continuous on \([a, b]\), and \( f'(t) \) and \( g'(t) \) are not simultaneously zero on \([a, b]\).)

\subsection*{The arc length \( L \)(曲线的长度 \( L \))}
- 对于函数 \( y = f(x) \)(\( a \leq x \leq b \)),弧长为
\[
L = \int_{a}^{b} \sqrt{1 + [f'(x)]^2} dx
\]
- 对于参数方程 \( x = f(t) \),\( y = g(t) \)(\( a \leq t \leq b \)),弧长为
\[
L = \int_{a}^{b} \sqrt{[f'(t)]^2 + [g'(t)]^2} dt
\]

\subsection*{Differential of Arc Length(曲线弧的微分)}
\[
ds = \sqrt{1 + \left( \frac{dy}{dx} \right)^2} dx \quad \text{或} \quad ds = \sqrt{\left( \frac{dx}{dt} \right)^2 + \left( \frac{dy}{dt} \right)^2} dt
\]

\subsection*{Area of a Surface of Revolution(旋转体的表面积)}
如果所给曲线的参数方程形式为 \( x = f(t) \),\( y = g(t) \),\( a \leq t \leq b \),那么表面积公式就可以表示为
\[
S = 2\pi \int_{a}^{b} g(t) \sqrt{[f'(t)]^2 + [g'(t)]^2} dt \quad (\text{绕x轴旋转})
\]
或
\[
S = 2\pi \int_{a}^{b} f(t) \sqrt{[f'(t)]^2 + [g'(t)]^2} dt \quad (\text{绕y轴旋转})
\]
(If the curve is given parametrically by \( x = f(t) \), \( y = g(t) \), \( a \leq t \leq b \), then the surface area formula becomes the above expressions for revolution about x-axis or y-axis.)

---

\section*{6.7 Probability and Random Variables}
\addcontentsline{toc}{section}{6.7 概率和随机变量}

如果 \( A \) 是一事件,那么有一组结果的可能性,我们记 \( A \) 的可能性为 \( P(A) \)。可能性具有下列性质:
1. 对于任意 \( A \),\( 0 \leq P(A) \leq 1 \)
2. 若 \( S \) 是所有结果可能的集合,那它就叫样本空间,\( P(S) = 1 \)
3. 若事件 \( A \) 和 \( B \) 是互不相容事件,就是说,两者之间没有共同点,那么 \( P(A \cup B) = P(A) + P(B) \)(事实上,需要更严格的条件,但现在足够了)

(If \( A \) is an event, that is, a set of possible outcomes, then we denote the probability of \( A \) by \( P(A) \). Probabilities must satisfy the following properties: 1. \( 0 \leq P(A) \leq 1 \) for every event \( A \); 2. If \( S \) is the set of all possible outcomes, called the sample space, then \( P(S) = 1 \); 3. If events \( A \) and \( B \) are disjoint, that is, they have no outcomes in common, then \( P(A \cup B) = P(A) + P(B) \). (Actually, a stronger condition is required, but for now this will do.))

\subsection*{Definition Expectation of a Random Variable(随机变量的期望)}
如果 \( X \) 是一个随机变量,有着如下概率分布:

\begin{center}
\begin{tabular}{c|cccc}
\hline
$ X $ & $ x_1 $ & $ x_2 $ & $ \cdots $ & $ x_n $ \\
\hline
$ P(X = x_i) $ & $ p_1 $ & $ p_2 $ & $ \cdots $ & $ p_n $ \\
\hline
\end{tabular}
\end{center}

则 \( X \) 的期望,记作 \( E(X) \),也称为 \( X \) 的均值且被记为 \( \mu \),是
\[
E(X) = x_1 p_1 + x_2 p_2 + \cdots + x_n p_n
\]
(If \( X \) is a random variable with probability distribution as above, then the expectation of \( X \), denoted \( E(X) \), also called the mean of \( X \) and denoted \( \mu \), is \( E(X) = x_1 p_1 + x_2 p_2 + \cdots + x_n p_n \).)

\subsection*{the probability density function (PDF)(概率密度函数)}
对于一个连续随机变量 \( X \),我们必须详细说明概率密度函数(PDF)。概率密度函数是指随机变量 \( X \) 在区间 \([A, B]\) 上的一个函数,它满足:
1. \( f(x) \geq 0 \) 对所有 \( x \in [A, B] \) 成立
2. \( \int_{A}^{B} f(x) dx = 1 \)
3. 对所有 \( a, b \in [A, B] \)(\( a \leq b \)),有 \( P(a \leq X \leq b) = \int_{a}^{b} f(x) dx \)

(For a continuous random variable \( X \), we must specify the probability density function (PDF). A PDF for a random variable \( X \) that takes on values in the interval \([A, B]\) is a function satisfying: 1. \( f(x) \geq 0 \) for all \( x \) in \([A, B]\); 2. \( \int_{A}^{B} f(x) dx = 1 \); 3. \( P(a \leq X \leq b) = \int_{a}^{b} f(x) dx \) for all \( a, b \) ($ a \leq b $) in the interval \([A, B]\).)

\subsection*{The cumulative distribution function (CDF)(累计分布函数)}
关于随机变量 \( X \) 的累积分布函数,定义如下:
\[
F(x) = P(X \leq x)
\]
这个函数是同时为离散和连续随机变量定义的。
(The cumulative distribution function (CDF), which, for a random variable \( X \), is the function \( F \) defined by \( F(x) = P(X \leq x) \). This function is defined for both discrete and continuous random variables.)

\subsection*{Theorem A}
令 \( X \) 是在区间 \([A, B]\) 上取值的连续随机变量,它的概率密度函数是 \( f(x) \),累积分布函数是 \( F(x) \),则
\[
F'(x) = f(x) \quad \text{对所有} \quad x \in (A, B)
\]
(Let \( X \) be a continuous random variable taking on values in the interval \([A, B]\) and having PDF \( f(x) \) and CDF \( F(x) \). Then \( F'(x) = f(x) \) for all \( x \) in \((A, B)\).)

---

\section*{7.1 Basic Integration Rules}
\addcontentsline{toc}{section}{7.1 基本积分规则}

\subsection*{Standard Integral Forms(标准积分形式)}
- Constants(常函数):\(\int k dx = kx + C\)
- Powers(幂函数):\(\int x^n dx = \frac{x^{n+1}}{n+1} + C\)(\( n \neq -1 \))
- Trigonometric Functions(三角函数):
  \[
  \int \sin x dx = -\cos x + C, \quad \int \cos x dx = \sin x + C, \quad \int \sec^2 x dx = \tan x + C
  \]
  \[
  \int \csc^2 x dx = -\cot x + C, \quad \int \sec x \tan x dx = \sec x + C, \quad \int \csc x \cot x dx = -\csc x + C
  \]
- Algebraic Functions(代数函数):\(\int \frac{1}{x} dx = \ln|x| + C\)

\subsection*{Theorem A Substitution in Indefinite Integrals(不定积分的换元积分法)}
令 \( g \) 为一个可导函数且函数 \( f \) 为函数 \( F \) 的导数,若 \( u = g(x) \),则
\[
\int f(g(x)) g'(x) dx = F(g(x)) + C
\]
(Let \( g \) be a differentiable function and suppose that \( F \) is an antiderivative of \( f \). Then, if \( u = g(x) \), \(\int f(g(x)) g'(x) dx = F(g(x)) + C\).)

\subsection*{Substitution in definite Integrals(定积分的换元积分法)}
look at 5.4 theorem C(见5.4定理C)

---

\section*{7.2 Integration by Parts}
\addcontentsline{toc}{section}{7.2 分部积分法}

\subsection*{Integration by Parts: Indefinite Integrals(不定积分的分部积分法)}
\[
\int u dv = uv - \int v du
\]
其中 \( u \) 和 \( v \) 是关于 \( x \) 的可导函数。
(where \( u \) and \( v \) are differentiable functions of \( x \).)

\subsection*{Integration by Parts: Definite Integrals(定积分的分部积分法)}
\[
\int_{a}^{b} u dv = \left. uv \right|_{a}^{b} - \int_{a}^{b} v du
\]

\subsection*{Reduction Formulas(递推公式)}
形如 \( \int f^n(x) dx = F(x, f^k(x)) + C \)(其中 \( k < n \))的式子,称作递推公式(\( f \) 的指数被降低)。
(A formula of the form \( \int f^n(x) dx = F(x, f^k(x)) + C \) where \( k < n \), is called a reduction formula (the exponent on \( f \) is reduced).)

PS: look at page393 example7、8(见书本P393例7、例8)

---

\section*{7.3 Some Trigonometric Integrals}
\addcontentsline{toc}{section}{7.3 三角函数的积分}

\subsection*{five commonly encountered types(五种最常见的类型)}
1. \( \int \sin^n x dx \) 和 \( \int \cos^n x dx \)
2. \( \int \sin^m x \cos^n x dx \)
3. \( \int \sin mx \cos nx dx \),\( \int \sin mx \sin nx dx \),\( \int \cos mx \cos nx dx \)
4. \( \int \tan^n x dx \),\( \int \cot^n x dx \)
5. \( \int \tan^n x \sec^m x dx \),\( \int \cot^n x \csc^m x dx \)

\subsection*{type1: \( \int \sin^n x dx \) 和 \( \int \cos^n x dx \)}
- 当 \( n \) 为正奇数时,在提出一个 \( \sin x \) 或 \( \cos x \) 后,利用 \( \sin^2 x + \cos^2 x = 1 \) 这个性质求解。
- 当 \( n \) 是偶数时,利用半角公式:
  \[
  \sin^2 x = \frac{1 - \cos 2x}{2}, \quad \cos^2 x = \frac{1 + \cos 2x}{2}
  \]
(Consider first the case where \( n \) is an odd positive integer. After taking out either the factor \( \sin x \) or \( \cos x \), use the identity \( \sin^2 x + \cos^2 x = 1 \). Make use of half-angle identities where \( n \) is even.)

\subsection*{type2: \( \int \sin^m x \cos^n x dx \)}
如果 \( m \) 和 \( n \) 中有一个是正奇数,而另一个是任意数,可以提出一个 \( \sin x \) 或 \( \cos x \),接着利用 \( \sin^2 x + \cos^2 x = 1 \) 这个性质求解。
(If either \( m \) or \( n \) is an odd positive integer and the other exponent is any number, we factor out \( \sin x \) or \( \cos x \) and use the identity \( \sin^2 x + \cos^2 x = 1 \).)

\subsection*{type3: 积化和差积分}
运用积化和差公式:
\[
\sin A \cos B = \frac{1}{2}[\sin(A+B) + \sin(A-B)]
\]
\[
\sin A \sin B = \frac{1}{2}[\cos(A-B) - \cos(A+B)]
\]
\[
\cos A \cos B = \frac{1}{2}[\cos(A+B) + \cos(A-B)]
\]
(use the product identities: the above equalities.)

\subsection*{type4: \( \int \tan^n x dx \) 和 \( \int \cot^n x dx \)}
对于 \( n \geq 2 \),被积函数是正切情况下,利用公式 \( \tan^2 x = \sec^2 x - 1 \);被积函数是余切情况下,利用公式 \( \cot^2 x = \csc^2 x - 1 \)。
(For \( n \geq 2 \), in the tangent case, factor out \( \tan^2 x = \sec^2 x - 1 \); in the cotangent case, factor out \( \cot^2 x = \csc^2 x - 1 \).)

\subsection*{type5: \( \int \tan^n x \sec^m x dx \) 和 \( \int \cot^n x \csc^m x dx \)}
- \( n \) 为偶数,\( m \) 为任意数,转化 \( \sec x \)(利用 \( \sec^2 x = 1 + \tan^2 x \))
- \( m \) 为奇数,\( n \) 为任意数,转化 \( \tan x \)(提出 \( \sec x \tan x \))
(n even, m Any number, transform \( \sec x \) (use \( \sec^2 x = 1 + \tan^2 x \)); m odd, n Any number, transform \( \tan x \) (factor out \( \sec x \tan x \)).)

---

\section*{7.4 Rationalizing Substitutions}
\addcontentsline{toc}{section}{7.4 第二类换元积分法}

\subsection*{Integrands Involving \( \sqrt[n]{ax + b} \)(被积函数涉及 \( \sqrt[n]{ax + b} \))}
如果 \( \sqrt[n]{ax + b} \) 出现在被积函数中,用 \( u = \sqrt[n]{ax + b} \) 换元可以解决问题。
(If \( \sqrt[n]{ax + b} \) appears in an integral, the substitution \( u = \sqrt[n]{ax + b} \) will eliminate the radical.)

\subsection*{Integrands Involving \( \sqrt{a^2 - x^2} \)、\( \sqrt{a^2 + x^2} \) 和 \( \sqrt{x^2 - a^2} \)(被积函数涉及以上三类的积分)}
为了有理化这些表达式,我们可以假设 \( a \) 是正数并作如下三角变换:

\begin{center}
\begin{tabular}{c|c|c}
\hline
Radical(根式) & Substitution(换元) & Restriction on \( t \)(限制\( t \)的取值范围) \\
\hline
\( \sqrt{a^2 - x^2} \) & \( x = a \sin t \) & \( -\frac{\pi}{2} \leq t \leq \frac{\pi}{2} \) \\
\( \sqrt{a^2 + x^2} \) & \( x = a \tan t \) & \( -\frac{\pi}{2} < t < \frac{\pi}{2} \) \\
\( \sqrt{x^2 - a^2} \) & \( x = a \sec t \) & \( 0 \leq t < \frac{\pi}{2} \) 或 \( \pi \leq t < \frac{3\pi}{2} \) \\
\hline
\end{tabular}
\end{center}

\subsection*{Completing the Square(完全平方法)}
当一个形如 \( x^2 + Bx + C \) 的二次式出现在根号下时,可以把它拼凑成完全平方式,以开始三角换元。
(When a quadratic expression of the type \( x^2 + Bx + C \) appears under a radical, completing the square will prepare it for a trigonometric substitution.)

---

\section*{7.5 Integration of Rational Functions Using Partial Fractions}
\addcontentsline{toc}{section}{7.5 用部分分式法求有理函数的积分}

\subsection*{different types of functions using partial fractions(用部分分式法的不同类型的式子)}
- distinct linear factors(不同线性因式)
- repeated linear factors(线性重因式)
- Some Distinct, Some Repeated Linear Factors(部分线性单因式,部分线性重因式)
- single quadratic factor(二次单因式)
- repeated quadratic factor(二次重因式)

(关注7.5 例题1-8)(see examples 1-8 in section 7.5)

\subsection*{Summary of partial fractions(关于部分分式的总结)}
分解一个有理函数 \( f(x) = \frac{p(x)}{q(x)} \) 成部分分式的过程如下:
\begin{enumerate}[label=Step \arabic*:]
    \item 如果 \( f(x) \) 为假分式,即如果 \( p(x) \) 的次方至少等于 \( q(x) \),用 \( p(x) \) 除以 \( q(x) \) 得 \( f(x) = s(x) + \frac{r(x)}{q(x)} \),其中 \( s(x) \) 是多项式且 \( \deg(r) < \deg(q) \)。
    (If \( f(x) \) is improper, that is, if \( p(x) \) is of degree at least that of \( q(x) \), divide \( p(x) \) by \( q(x) \), obtaining \( f(x) = s(x) + \frac{r(x)}{q(x)} \), where \( s(x) \) is a polynomial and \( \deg(r) < \deg(q) \).)
    \item 将 \( q(x) \) 分解成实系数的线性和不可约的二次因式:由代数定理可知,(理论上)这是绝对可能的。
    (Factor \( q(x) \) into a product of linear and irreducible quadratic factors with real coefficients. By a theorem of algebra, this is always (theoretically) possible.)
    \item 对于每个 \( (ax + b)^k \) 形式的因式,可以期待分解成:
      \[
      \frac{A_1}{ax + b} + \frac{A_2}{(ax + b)^2} + \cdots + \frac{A_k}{(ax + b)^k}
      \]
      其中 \( A_1, A_2, \cdots, A_k \) 是常数。
      (For each factor of the form \( (ax + b)^k \), expect the decomposition to have the terms above where \( A_1, A_2, \cdots, A_k \) are constants.)
    \item 对于每个 \( (ax^2 + bx + c)^m \) 形式的因式,可以期待分解成:
      \[
      \frac{B_1x + C_1}{ax^2 + bx + c} + \frac{B_2x + C_2}{(ax^2 + bx + c)^2} + \cdots + \frac{B_mx + C_m}{(ax^2 + bx + c)^m}
      \]
      其中 \( B_1, C_1, \cdots, B_m, C_m \) 是常数。
      (For each factor of the form \( (ax^2 + bx + c)^m \), expect the decomposition to have the terms above where \( B_1, C_1, \cdots, B_m, C_m \) are constants.)
    \item 令 \( \frac{r(x)}{q(x)} \) 等于第三步和第四步中所有部分的和。常数的数目要等于分母 \( q(x) \) 的次数。
    (Set \( \frac{r(x)}{q(x)} \) equal to the sum of all the terms found in Steps 3 and 4. The number of constants to be determined should equal the degree of the denominator \( q(x) \).)
    \item 将第五步建立的等式的两边都乘以 \( q(x) \) 并求出未知常数。这可以用两种方法算出:
      (1) 令相同指数部分的系数相等;或 (2) 令 \( x \) 等于简单的值。
      (Multiply both sides of the equation found in Step 5 by \( q(x) \) and solve for the unknown constants. This can be done by either of two methods: (1) Equate coefficients of like-degree terms; or (2) assign convenient values to the variable \( x \).)
\end{enumerate}

\subsection*{the logistic differential equation(罗辑斯蒂微分方程)}
\[
y' = ky(L - y)
\]
(关注p411 例题9)(see example 9 on page 411)

---

\section*{7.7 First-Order Linear Differential Equations}
\addcontentsline{toc}{section}{7.7 一阶线性微分方程}

\subsection*{a first-order linear differential equation(一个一阶线性微分方程)}
\[
\frac{dy}{dx} + P(x)y = Q(x)
\]
其中 \( P(x) \) 和 \( Q(x) \) 仅仅是 \( x \) 的函数。
(where \( P(x) \) and \( Q(x) \) are functions of \( x \) only.)

\subsection*{Solving First-Order Linear Equations(解决一阶线性方程)}
为了求解一阶线性微分方程,我们首先将两边都乘以积分因子
\[
\mu(x) = e^{\int P(x) dx}
\]
(采取这一步骤的原因很快就会变得清楚。)微分方程变为
\[
e^{\int P(x) dx} \frac{dy}{dx} + P(x) e^{\int P(x) dx} y = Q(x) e^{\int P(x) dx}
\]
左边是乘积 \( y \cdot e^{\int P(x) dx} \) 的导数,所以这个方程的形式是
\[
\frac{d}{dx} \left( y e^{\int P(x) dx} \right) = Q(x) e^{\int P(x) dx}
\]
双方整合后
\[
y e^{\int P(x) dx} = \int Q(x) e^{\int P(x) dx} dx + C
\]
通解是
\[
y = e^{-\int P(x) dx} \left( \int Q(x) e^{\int P(x) dx} dx + C \right)
\]

没有必要记住最后结果,求解过程容易推导。
(It is not worth memorizing this final result; the process of getting there is easily recalled and that is what we illustrate.)

(ps: 关注423 例题1、2 辅助理解)(see examples 1 and 2 on page 423 for better understanding)

---

\section*{8.1 Improper Integrals: Infinite Limits of Integration}
\addcontentsline{toc}{section}{8.1 反常积分:无穷区间上的反常积分}

\subsection*{Definition(定义)}
- 对于 \( \int_{a}^{\infty} f(x) dx \):
  \[
  \int_{a}^{\infty} f(x) dx = \lim_{b \to \infty} \int_{a}^{b} f(x) dx
  \]
- 对于 \( \int_{-\infty}^{b} f(x) dx \):
  \[
  \int_{-\infty}^{b} f(x) dx = \lim_{a \to -\infty} \int_{a}^{b} f(x) dx
  \]
- 对于 \( \int_{-\infty}^{\infty} f(x) dx \):
  \[
  \int_{-\infty}^{\infty} f(x) dx = \int_{-\infty}^{c} f(x) dx + \int_{c}^{\infty} f(x) dx
  \]
  其中 \( c \) 为某个常数。

如果等式右边的极限存在而且是一个有限的数,那么我们就说对应的反常积分收敛且收敛于这个数;否则,我们就说这些反常积分发散。
(If the limits on the right exist and have finite values, then we say that the corresponding improper integrals converge and have those values. Otherwise, the integrals are said to diverge.)

\end{document}